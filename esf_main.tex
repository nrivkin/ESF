\documentclass{article}
\usepackage[margin=1in, paperwidth=8.5in, paperheight=11in]{geometry}
\usepackage{caption}
\usepackage{subcaption}
\usepackage{chngcntr}
\usepackage{float}
\usepackage{graphicx}
\usepackage[utf8]{inputenc}


\counterwithin{table}{section} %number tables with section#.table#


\begin{document}
\section*{Table of Contents iii}
TODO

\section*{I List of Figures}
TODO

\section*{II List of Tables}
TODO

\section*{III List of Abbreviations}
TODO

\section*{1 System Overview}
TODO

\section*{2 Electrical Systems}
TODO

\subsection*{2.1 Shutdown Circuit}
TODO

\subsubsection*{2.1.1 Description/concept}
TODO

\subsubsection*{2.1.2 Wiring / additional circuitry}
TODO

\subsubsection*{2.1.3 Position in car}
TODO

\subsection*{2.2 IMD}
\subsubsection*{2.2.1 "Description (type, operation parameters)"}

TODO
The IMD used will be a Bender A-ISOMETER IR155-3204. The output is normally high and only low if it does not detect a ground fault. The output is then sent to the IMD latch board, where it is powers a SPST-NO relay which closes a switch in the shutdown circuit. The output of the IMD is also monitored directly by an Atmega on the AIR control board. The status of the shutdown circuit is monitored by the ???? board. If the ???? board detects the relay is  open it activates the IMD light.


%Describe the IMD used and use a table for the common operation parameters, like supply voltage, set point, etc. Also describe how the IMD indicator light is wired, etc.


\begin{center}
	\begin{table}[H]
		\begin{tabular}{|l|l|}
			\hline
			Supply voltage range: &  10..36VDC \\
			\hline
			Supply voltage: &  12VDC\\
			\hline
			Environmental temperature range: &  -40..105$^{\circ}$C \\
			\hline
			Selftest interval: &  Always at startup, then every 5 minutes \\
			\hline
			High voltage range: &  DC 0..1000V \\
			\hline
			Set response value: &  100k$\Omega$ \\
			\hline
			Max. operation current: &  150mA \\
			\hline
			Approximate time to shut down at 50\% of the response value:&  $\leq$ 40s \\
			\hline
		\end{tabular}
		\caption{Parameters of the IMD}
		\label{IMDParameters}
	\end{table}
\end{center}







\subsubsection*{2.2.2 Wiring/cables/connectors/}
TODO

The connector on the IMD is the TYCO-MICRO MATE-N-LOK 1 x 2-1445088-8. It is attached to a 28 AWG ribbon cable that is connected to the IMD latch using a 6-pin TE Connectivity AMP connector. The 28 AWG wire is used because the current draw of the IMD is 150mA, which allows high gauge wire to be used.

%Describe wiring, show schematics, describe connectors and cables used and show useful data regarding the wiring including wire gauge/temp/voltage rating and fuses protecting the wiring.

% I don't know if the IMD has a built in fuse

\subsubsection*{2.2.3 Position in car}
%TODO
%Position in car
%Provide CAD-renderings showing the relevant parts. Mark the parts in the rendering, if necessary.

The IMD will be located inside the accumulator, as shown in Section 3.1.. This is a convenient location for the IMD as high voltage sensing lines must already be present here for the TSMP’s.


\subsection*{2.3 Inertia Switch}
TODO

\subsubsection*{2.3.1 "Description (type, operation parameters)"}
TODO

\subsubsection*{2.3.2 Wiring/cables/connectors/}
TODO

\subsubsection*{2.3.3 Position in car}
TODO

\subsection*{2.4 Brake Plausibility Device}
TODO

\subsubsection*{2.4.1 Description/additional circuitry}
TODO

\subsubsection*{2.4.2 Wiring}
TODO

\subsubsection*{2.4.3 Position in car/mechanical fastening/mechanical connection}
TODO

\subsection*{2.5 Reset / Latching for IMD and BMS}
\subsubsection*{2.5.1 Description/circuitry}
%Describe the concept and circuitry of the latching/reset system for a tripped IMD or BMS.  Describe the method for resetting the IMD and BMS.
TODO

\subsubsection*{2.5.2 Wiring/cables/connectors}
%Describe wiring, show schematics, describe connectors and cables used and show useful data regarding the wiring.  If not detailed in section 2.1, be sure to show how the device opens the shutdown circuit.
TODO

\subsubsection*{2.5.3 Position in car}
%Provide CAD-renderings showing the relevant parts. Mark the parts in the rendering, if necessary.
TODO

\subsection*{2.6 Shutdown System Interlocks}
TODO

\subsubsection*{2.6.1 Description/circuitry}
TODO

\subsubsection*{2.6.2 Wiring/cables/connectors}
TODO

\subsubsection*{2.6.3 Position in car}
TODO

\subsection*{2.7 Tractive system active light}
TODO

\subsubsection*{2.7.1 Description/circuitry}
TODO

\subsubsection*{2.7.2 Wiring/cables/connectors}
TODO

\subsubsection*{2.7.3 Position in car}
TODO

\subsection*{2.8 Measurement points}
TODO

\subsubsection*{2.8.1 Description}
TODO

\subsubsection*{2.8.2 "Wiring, connectors, cables"}
TODO

\subsubsection*{2.8.3 Position in car}
TODO

\subsection*{2.9 Pre-Charge circuitry}
TODO

\subsubsection*{2.9.1 Description}
TODO

\subsubsection*{2.9.2 "Wiring, cables, current calculations, connectors"}
TODO

\subsubsection*{2.9.3 Position in car}
TODO

\subsection*{2.1 Discharge circuitry}
TODO

\subsubsection*{2.10.1 Description}
TODO

\subsubsection*{2.10.2 "Wiring, cables, current calculations, connectors"}
TODO

\subsubsection*{2.10.3 Position in car}
TODO

\subsection*{2.11 HV Disconnect (HVD)}
TODO

\subsubsection*{2.11.1 Description}
TODO

\subsubsection*{2.11.2 "Wiring, cables, current calculations, connectors"}
TODO

\subsubsection*{2.11.3 Position in car}
TODO

\subsection*{2.12 Ready-To-Drive-Sound (RTDS)}
TODO

\subsubsection*{2.12.1 Description}
TODO

\subsubsection*{2.12.2 "Wiring, cables, current calculations, connectors"}
TODO

\subsubsection*{2.12.3 Position in car}
TODO

\section*{3 Accumulator}
TODO

\subsection*{3.1 Accumulator pack 1}
TODO

\subsubsection*{3.1.1 Overview/description/parameters}
TODO

\subsubsection*{3.1.2 Cell description}
TODO

\subsubsection*{3.1.3 Cell configuration}
TODO

\subsubsection*{3.1.4 Cell temperature monitoring}
TODO

\subsubsection*{3.1.5 Battery management system}
TODO

\subsubsection*{3.1.6 Accumulator indicator}
TODO

\subsubsection*{3.1.7 "Wiring, cables, current calculations, connectors"}
TODO

\subsubsection*{3.1.8 Accumulator insulation relays}
TODO

\subsubsection*{3.1.9 Fusing}
TODO

\subsubsection*{3.1.10 Charging}
TODO

\subsubsection*{3.1.11 Mechanical Configuration/materials}
TODO

\subsubsection*{3.1.12 Position in car}
TODO

\subsection*{3.2 Accumulator pack 2}
TODO

\section*{4 Energy meter mounting}
TODO

\subsection*{4.1 Description}
TODO

\subsection*{4.2 "Wiring, cables, current calculations, connectors"}
TODO

\subsection*{4.3 Position in car}
TODO

\section*{5 Motor controller}
TODO

\subsection*{5.1 Motor controller 1}
TODO

\subsubsection*{5.1.1 "Description, type, operation parameters"}
TODO

\subsubsection*{5.1.2 "Wiring, cables, current calculations, connectors"}
TODO

\subsubsection*{5.1.3 Position in car}
TODO

\subsection*{5.2 Motor controller 2}
TODO

\section*{6 Motors}
TODO

\subsection*{6.1 Motor 1}
TODO

\subsubsection*{6.1.1 "Description, type, operating parameters"}
TODO

\subsubsection*{6.1.2 "Wiring, cables, current calculations, connectors"}
TODO

\subsubsection*{6.1.3 Position in car}
TODO

\subsection*{6.2 Motor 2}
TODO

\section*{7 Torque encoder}
TODO

\subsection*{7.1 Description/additional circuitry}
TODO

\subsection*{7.2 Torque Encoder Plausibility Check}
TODO

\subsection*{7.3 Wiring}
TODO

\subsection*{7.4 Position in car/mechanical fastening/mechanical connection}
TODO

\section*{8 Additional LV-parts interfering with the tractive system}
TODO

\subsection*{8.1 LV part 1}
TODO

\subsubsection*{8.1.1 Description}
TODO

\subsubsection*{8.1.2 "Wiring, cables,"}
TODO

\subsubsection*{8.1.3 Position in car}
TODO

\subsection*{8.2 LV part 2}
TODO

\section*{9 Overall Grounding Concept}
TODO

\subsection*{9.1 Description of the Grounding Concept}
TODO

\subsection*{9.2 Grounding Measurements}
TODO

\section*{10 Firewall(s)}
TODO

\subsection*{10.1 Firewall 1}
TODO

\subsubsection*{10.1.1 Description/materials}
TODO

\subsubsection*{10.1.2 Position in car}
TODO

\subsection*{10.2 Firewall 2}
TODO

\section*{11 Appendix}
TODO

\end{document}
TODO
